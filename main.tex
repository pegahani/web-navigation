%
% File naacl2019.tex
%
%% Based on the style files for ACL 2018 and NAACL 2018, which were
%% Based on the style files for ACL-2015, with some improvements
%%  taken from the NAACL-2016 style
%% Based on the style files for ACL-2014, which were, in turn,
%% based on ACL-2013, ACL-2012, ACL-2011, ACL-2010, ACL-IJCNLP-2009,
%% EACL-2009, IJCNLP-2008...
%% Based on the style files for EACL 2006 by 
%%e.agirre@ehu.es or Sergi.Balari@uab.es
%% and that of ACL 08 by Joakim Nivre and Noah Smith

\documentclass[11pt,a4paper]{article}
\usepackage[hyperref]{naaclhlt2019}
\usepackage{times}
\usepackage{latexsym}

\usepackage{url}

%\aclfinalcopy % Uncomment this line for the final submission
%\def\aclpaperid{***} %  Enter the acl Paper ID here

%\setlength\titlebox{5cm}
% You can expand the titlebox if you need extra space
% to show all the authors. Please do not make the titlebox
% smaller than 5cm (the original size); we will check this
% in the camera-ready version and ask you to change it back.

\newcommand\BibTeX{B{\sc ib}\TeX}

%%%%%neww added commands by writer%%%%%%
\newcommand{\PA}[1]{{\textcolor{blue}{#1}}}
\newcommand{\question}[1]{{\textcolor{orange}{#1}}}

\definecolor{rouge}{rgb}{1,0,0}
\newcommand{\remJGF}[1]{{\color{rouge}{\emph{JGF : #1}}}}

\renewcommand{\labelitemii}{$\diamond$}

\usepackage{amssymb}
\usepackage{amsmath}
\usepackage{amsfonts}
\usepackage{graphicx}
\usepackage[ruled,linesnumbered]{algorithm2e}

%%%%%neww added commands by writer%%%%%%

%\title{A Navigation Web Search System for  Expert Tracking \\ Using Deep Reinforcement Learning}
\title{Deep Reinforcement Learning for Expert Mobility Tracking from Web Search Results}

\author{Pegah Alizadeh \\
  Affiliation / Address line 1 \\
  Affiliation / Address line 2 \\
  Affiliation / Address line 3 \\
  {\tt email@domain} \\\And
  Josue Urbina \\
  Affiliation / Address line 1 \\
  Affiliation / Address line 2 \\
  Affiliation / Address line 3 \\
  {\tt email@domain}
  Carl Posthuma \\
  Affiliation / Address line 1 \\
  Affiliation / Address line 2 \\
  Affiliation / Address line 3 \\
  {\tt email@domain} 
  Jorge Garcia Flores \\
  Affiliation / Address line 1 \\
  Affiliation / Address line 2 \\
  Affiliation / Address line 3 \\
  {\tt email@domain} 
  Ivan Meza \\
  Affiliation / Address line 1 \\
  Affiliation / Address line 2 \\
  Affiliation / Address line 3 \\
  {\tt email@domain} 
  Luis Pineda \\
  Affiliation / Address line 1 \\
  Affiliation / Address line 2 \\
  Affiliation / Address line 3 \\
  {\tt email@domain}
  \\}
\date{}

\begin{document}
\maketitle
\begin{abstract}

  This paper presents an original approach for extracting experts mobility traces from the web based on deep reinforcement learning. In the broad context of Digital Humanities, the goal of our work is to assist migration sociologists studying expert's international mobility trajectories with natural language processing methods. Deep-Q networks have recently shown promising results improving information extraction methods from web results. Given an expert's name, our method harvests information from web search engines and extracts the optimal expert's mobility trajectory (a set of triplets containing institution, city and year) from it by means of a Deep-Q network and various neural network architectures for the Q-value function approximation. This paper presents experimental evidence improving the current state of the art on expert finding, and extending the existing task with a chronological graph of the experts professional trajectory. 

  %We present at a first time, a people tracking system (specifically the expert people with available and extractable information in the web) as an application of deep reinforcement learning and natural language processing methods to the immigration sociology. The Deep Q-network has been used for extracting information from huge size text datasets and finding the most precise name entities. In this paper, for the sociology term of usage, we propose a method for tracking a give person name and extracting her optimal professional trajectory by exploring the web and generating as less as possible number of queries. Our proposed approach is based on a Deep Q-network method with various neural networks for the Q-value function approximation. We improve the existed people search systems by tracking the experts professional trajectories instead of just labeling them as a mobile or local person w.r.t their origin countries.  
\end{abstract}

\section{Introduction}
Luis Fernandez\footnote{All names have been changed to protect identity.} grew up in Argentina and studied a Biotechnology Masters's degree at the University of Buenos Aires. In $2010$ he got an international scholarship and went to Essex University for a PhD. Where is he now? Did he stay in the UK, or did he come back to Argentina? How did his expert skills evolve? In other words, what are his \textit{professional mobility traces}?

Reaching expatriate members of the \textit{highly qualified diaspora} is a major challenge for policy makers \cite{CIDESAL}. Migrations sociologists studying mobility patterns of this \textit{diaspora of knowledge} use statistical sources, like population censues, labor force surveys and administrative data \cite{Turner}. However, these sources are not entirely satisfactory when monitoring constantly changing mobility trends, so recent scientific efforts have been dedicated to mine the Web in order better understand the dynamics of this expert migration flow \cite{GarciaFlores, Russians, Auriol2010}.

%%%%%%%%%%%
\begin{figure}[!t]
\centering
\includegraphics[scale=0.3]{./images/trajectory.png}
\caption{Mobility trajectories of two Biotechnology experts.}
\label{fig:traj}
\end{figure}
%%%%%%%%%
This paper presents ongoing research on extracting experts mobility trajectories from the web by means of a reinforcement learning (RL}) based method. Given an expert's name, our method harvests information from web search engines in order to extract a set of named entity triplets \texttt{<institution, city, year>} considered as meaningful points in the expert's mobility trajectory. Figure~\ref{fig:traj} shows the mobility trajectory of two biotechnology experts. Migration sociologists are particularly interested in the \textit{brain drain/brain gain} phenomenon \cite{Meyer}: while the top example of the figure has not returned to his/hers home country, the bottom individual has.
%In this work, we are interested in extracting a trajectory for each given expert including her organisation names and the related years by searching their given names through the web and extracting as accurate as possible name entities (In the rest of this work, by the name entities we mean the organisation names, university names or dates) from the web.  For instance, a researcher as ``Luis Fernandez" has a mobile professional trajectory with various oraganisations and year of experience as figure~\ref{fig:traj} (top graph) though another person (the bottom graph) has a local professional trajectory and he works in his original country recently.  
  
An intuitive approach for extracting experts mobility trajectories consists on querying web search engines with the expert's name plus different keywords, and then extracting city, institution and year from the resulting snippets list. The main question then would be when to generate a new query and which snippets should be taken into account in order to extract meaningful named entities concerning the expert's professional trajectory. Recent scientific evidence shows promising results of Deep-Q networks based methods on information extraction tasks \citep{narasimhan2016improving}, as well as on object detection in images \cite{Caicedo2015}, arithmetic word problems solving \cite{wang2018} and text generation \citep{Guo2015}. Inspired by Narasimhan et al. \shortcite{narasimhan2016improving} work on information extraction from huge search spaces, we model the expert's mobility traces extraction task as an optimal policy search in RL, where we consider a Deep Q-network and various neural network models to learn Q-value functions and approximate the optimal expert's track for a given expert name.  
%for tracking each given expert's profession usingbased  deep reinforcement learning (DRL) approach. Deep Q-network (DQN) usage in various problems and the tracking issue as a planning problem motivate us concentrating on this approach.

The main contribution of the present paper are:
\begin{itemize}
\item To the best of our knowledge, this is the first expert's mobility tracking approach based on  Deep RL methods. 
\item The Q-value function depends on the sequence of resulting snippets selection through from web search engines. For this reason, we replace the Deep-Q neural network \cite{mnih2015} with a Long Short Term Memory {\smallLSTM} network and \PA{improving previous results by \%  }
\item \PA{We test our methods on .... dataset and improve the results by .... }
\end{itemize}

%\begin{figure}[!t]
%\centering
%\includegraphics[scale=0.22]{./images/navigate.png}
%\caption{\PA{This figure motivates the usage of reinforcement learning in navigating web. But it has two problems: quality and confidential names. Even if I selected a common general Spanish name.}}
%\label{fig:navigate}
%\end{figure}

\section{Related Work}
\remJGF{Complete the Digital Humanities part}
\remJGF{While this task is close to Web People Search \cite{WPS}, here we choose a different approach, where we progressively query} 
In this section we review the literature of the \textit{Deep Reinforcement Learning (DRL)} methods used for the web based information extraction problems. We also introduce a background on the application of \textit{Natural Language Processing (NLP)} and DRL methods on sociology of migration application and Web People Search (WPS) tasks. 

\paragraph{Web People Search (\PA{or any similar title reviewing the literature for navigating immigration data base in the web})}

\paragraph{Deep Reinforcement Learning}
In the Reinforcement Learning (RL) environment with a state set $S = s_1, \cdots, s_m$ and a  set of possible actions $A = a_1, \cdots, a_m$, the agent interacts with the environment by selecting an action $a$ in each state $s$ and going to the next state $s'$ until the terminal state condition is satisfied. This process is guided by a function namely \textit{policy} $\pi : S \longrightarrow A$ by interacting the environment. The policy is calculable by assigning a punishment or reward $r(s, a)$ to each selected action $a$ in any state $s$ and maximizing the expected some of rewards \cite{sutton1998}.  

A way of defining the quality of policies is the Q-value function $Q : S \times A \longrightarrow \mathbb{R}$ given by: 
$$Q^{\pi}(s, a) = \mathbb{E}[R_1+ \gamma R_2 + \cdots | s_0 = s, a_0 =a, \pi]$$

which is the expectation of sum of rewards by starting in state $s$ taking action $a$ and following policy $\pi$. Here $\gamma \in (0, 1]$ is a discount factor for future rewards. Considering the Q-value of the \textit{optimal policy} as $Q^*$, then the optimal policy is: $\pi^*(s) = \text{argmax}_a Q^{\pi^*}(s, a) \; \forall  s \in S$.  

Q-learning algorithm \cite{sutton1998} is a model free method for learning the optimal Q-value function. Since Q-value function should be calculated for each sequence of state$\backslash$actions, computing $Q(s, a)$ for real environments with great number of states or actions  is impractical.  To tackle this problem Mnih et al.\shortcite{mnih2015} introduce the Deep Q-network for approximating the Q-value function with a non-linear multi layer convolutional neural network.  Given state $s$ and action $a$, the DQN gives $Q(s, a; \theta)$ value where $\theta$ are parameters for the neural network. (\question{If we model the Q-value function as $Q(s, .; \theta)$ where gets sate $s$ and outputs a vector of action values is better in term of calculation time. Our current model is the naive form.})




\section{Framework and Solution}
\paragraph{Queries: }  Since \textit{unoporuno} is an intelligent web navigation system, it requires some search engines and query generation models. In our work , we use four search engines including Google, DuckDuckGo, Researchgate, Citeceerx \PA{the search engine list should be modified. I didn't find the list.} . As person name is received from an available sociologist database, for each person we have $7$ possible number of queries as: \\

$<$person name$>$ $+$ (  $|$ doctorate $|$ institute $|$ master $|$ undergraduate $|$ university ) 

\paragraph{Markov Decision Process: } Using a similar framework as Narasimhan et al. \shortcite{narasimhan2016improving}, we model our web navigator model as a Markov Decision Process (MDP) \cite{puterman1994}. . .

This MDP is defined as a tuple $M(S, A, P, r, \gamma)$ where $S$ is a set of states, $A$ is a set of actions, $P :S\times A  \times S  \longrightarrow [0,1]$ is a transition function where $P(s'|s,a)$ encodes the probability of going to state $s'$ by being in state $s$, and choosing action $a$; $r : S \times A \longrightarrow R$ is a reward function (or penalty, if negative) obtained by choosing action $a$ in state $s$. Each parameter is detailed as the following.
\begin{itemize}
	\item The \textit{states} should be modeled in a way indicating the difference between the current and new snippets.  We use two models of sates in this paper (\question{I am wondering that there is no comparison between current and previous snippets in this model. I got the model from our code!}):
	\begin{itemize}
		\item \textit{Vector based}: each state indicates with a vector of real values containing: \\
		-- one-hot $7$ encoding indicating which query type is used for generating the snippet,\\
	    -- one-hot $4$ dimensional encoding indicating which research engine is used, \\
		-- number of common organisation names and dates between gold standards for a given person and extracted name entities in the new snippet,\\
        -- confident score for extracted organisation names and dates. 
		\item \textit{text based}: Each snippet is considered as a state ( \PA{for keeping this definition we should be certain about Ivan codes and results.} )
	\end{itemize}
	\item We have two \textit{Actions} including $q$ and $d$ indicating changing query and staying on the current query respectively. 
	\item The \textit{transition function} $P$ 
	\item The system goal is to extract the closest extracted organisation names and dates to the gold standards by generating less number of queries. For this reason, choosing $q$ action in any sate has a negative reward. And every time the extracted organisation name or date from a new snippet is close to the gold standards, the system receives a positive reward for instance $r(s, q) = -0.1$ or $r(s, d) = 10$ if the extracted organisation name (year) is in the list of organisation names (years) for gold standards. 
\end{itemize}



\section{Data}
We test the experiments on (\PA{three?}) different datasets.  The first dataset is a collected dataset including (\PA{how many persons in experiment?})  researchers and experts from central and sound America. A few information such as year, organisation, graduate degree, original country, domain and destination country of each person who left his/her original country for, is supported in the dataset prepared by the sociologists. 

Another dataset is \PA{Jorge dataset?}

Finally we experiment the navigation web system on \PA{shooting dataset with LSTM instead of NN in DQN}

\section{Experiments}

\section{Conclusions}

\section*{Acknowledgments}

\bibliography{naaclhlt2019}
\bibliographystyle{acl_natbib}

\appendix

\section{Appendices}
\label{sec:appendix}


\section{Supplemental Material}
\label{sec:supplemental}

\end{document}
